\documentclass[conference]{IEEEtran}
\IEEEoverridecommandlockouts
% The preceding line is only needed to identify funding in the first footnote. If that is unneeded, please comment it out.
\usepackage{cite}
\usepackage{amsmath,amssymb,amsfonts}
\usepackage{algorithmic}
\usepackage{graphicx}
\usepackage{textcomp}
\usepackage{xcolor}
\def\BibTeX{{\rm B\kern-.05em{\sc i\kern-.025em b}\kern-.08em
    T\kern-.1667em\lower.7ex\hbox{E}\kern-.125emX}}

\usepackage[style=ieee]{biblatex}
\addbibresource{ref.bib}

\begin{document}

\title{Recycling of Rare Earth Elements}

\author{\IEEEauthorblockN{Marcus Dahlström}
\IEEEauthorblockA{\textit{Chalmers University of Technology} \\
\textit{at Technische Universität Wien}\\
Vienna, Austria, 2025-01 \\
Student ID: 12402525}
}

\maketitle

\begin{abstract}
    Rare earth elements (REEs) are critical components in modern techonology. They play a role in everything from decarbonization to medical devices. This report examines what REEs are, where they are used, and how they are recycled. Focus is put on the environmental effects of aquisition though mining, which is mostly related to land disturbance and pollution, and the different processes of recycling, such as hydro- and pyrometallurgy. The geopolitical aspects of REEs, where China is the largest supplier, are also discussed. Lastly, there will be a look at the remaning supply of REEs and what alternative sources there are. The report concludes that .... TODO
\end{abstract}

\begin{IEEEkeywords}
rare earths elements, electronic waste, recycling
\end{IEEEkeywords}

\section{Background}

The following report on \textit{rare earth elements} (hereinafter referred to as \textit{REEs}), is aimed at individuals who are looking for an introduction to REEs from an environmental and recycling point of view. An introduction will be given to what REEs are, the different categories of REEs, the different properties that they have, where they are used, and why they are called rare. Following that chapter, there will be an overview of the global aspects of REEs, where they are mined, in which sectors they are used, and the different political aspects that are connected to them. After that comes the main two chapters of the report; namely the environmental aspects of acquisition, refinement, and their afterlife; followed by \textit{if} and \textit{how} they are recycled. The report will end with an overview of the future of REEs, specifically if there are any alternatives and what the main research topics are.

\subsection{Topics that will be left out}

As the main focus of the report is on the environmental and recycling aspects of REEs, the \textit{what} and \textit{why} of the chemical, mechanical, thermal, and electrical properties of REEs will be left out. Also, the ethical discussion will be held brief so as not to make the report too long.


\section{Introduction}
The main focus of this report will be on the environmental aspects and the recycling of REEs. The following chapter will therefore introduce REEs for individuals who are only vaguely or not at all familiar with REEs. As electronic waste, also called \textit{e-waste}, and recycling of REEs are tightly connected, there will also be a short introduction to the topic of e-waste. 

\subsection{What are REEs?}

Metals are usually divided into base, precious, and rare earth elements (src7). REEs are a group of 17 elements from the periodic table (src1). They were first discovered in the 18th century in a village called \textit{Ytterby} located in Sweden, by army lieutenant Carl Axel Arrhenius (src3). Element nr. 70 of the periodic table is called \textit{Ytterbium} for this reason. The first REE to be isolated was \textit{cerenium} in 1803 (src3).  

\subsection{Why they are called \textit{rare}}

REEs belong to the 50th percentile of most abundant elements in the earth's crust (src3). This is contrary to what one might expect with something called \textit{rare}. But the word \textit{rare} in \textit{rare earth elements} does not come from their rarity in the earth's crust, as is the case e.g. with diamonds (add src), but rather from the availability of the refined material (add src).  REEs are distributed throughout the earth's crust, but due to their geological(?) properties, they are dispersed mostly in low quantities, too low for there to be any economic value in mining and refining them (add src). The average quantity throughout the earth's crust is XXX (add src), while the lowest quantity that is mined purely for REEs is XXX (add erc). Most of the time, REEs have to be mined together with other minerals for there to be any economic value in mining them(add src). 

\subsection{Applications of REEs}

The use cases for REEs are wide, from the probably most well-known being neodymium magnets (src1) to lesser-known ones such as mirrors (src3). The many different magnetic (src1), electric (add src), thermal (add src), and catalytic (add src) properties make them usable in many areas. They have a role in applications within national security, as many modern defence systems rely on the use of REEs (src1). They are also used in the production of batteries, phosphors, and catalysts (src1). Their use in these low-level applications makes them a critical part of many modern technologies, in everything from healthcare and transportation to energy generation and consumer electronics (src1). In electronics, between 0.1-5\% of the total weight of the product is REEs. But for magnets, this can be as high as 25\% (src3). REEs have also found new uses in additive manufacturing, where they have been shown to increase the durability of the materials (src2). REEs can also be used in cement, where they improve durability and therefore reduce the need for repair and replacement (src2). REEs can be used in cleaning soil, in a similar way as they're used in cleaning exhaust gases, by binding to heavy metals and removing them from the soil (src2). For the same reason, they can also be used to filter water (src2).


\section{Global aspects}

The following chapter will cover the global aspects of REEs, such as the world wide yearly usage, economic value, global branch distribution, how and where they are mined, and political aspects. As the aquisition of REEs is a global issue, it is heaviliy dependent on geopolitical and  geoecomical factors (src2).

\subsection{World wide yearly usage and supply}

The use of REEs is many sectors and coutries. In the graph/table below is an overview (TODO). China is believed to have a black market for REEs, which is not included in the official numbers, but are believed to add an additional 25\% to their production (src 3). Demand for REEs increase at a rate of 10\% every year. Both the EU and the US have REEs on their lists of critical raw materials (src8 -> EU 2014, US 2018).

\subsection{Economic value}

For there to be any econmical value in mining for REEs, the concentration has to be 5\% or higher (src 3). If mined together with other minerals, it can be economically viable to mine down to 0.5\% concentrations (src 3). Advancements in aquisition and recycling is also influencing the price (src2). The price is influenced by export regulations, tariffs, and trade agreements (src2).

\subsection{Global branch distribution}

The push for decarbonisation has increased the demand for REEs, as they are a critical component in electric vehicles and wind turbines (src2).

\subsection{Where are they acquired?}

REEs are mostly aquired through mining (src3, src2). Secondary resource is recycling (src2). There are currently 160 mineral sources of REEs, however, only 4 of them are mined: bastnasite, laterite, monazite, and loparzite (src3). There are also other non-mineral sources of REEs, such as deep-sea clays outside of the Minamitori Islands in Japan, but they have not yet been exploited (src3). 80\% of the global supply is mined by China, mostly in the Inner Mongolia region (src 3). However, extraction takes place in many countries (src3). Other major deposits world wide are US, Canada, Australia, Russia, South Africe, south-east Asia, and India (src3).

\subsection{Political aspects}

Since China is the worlds largest producer of REEs, with 80\% (105'000 tons of RE oxide) of the global supply produced in 2017 (src3), and 70\% of the market share in 2024 (src2), there are reasons for mentioning the political aspects of REEs (src3). In the years 2009-2011 China could increase their profits on REEs by simply imposing export quotas and therefore drive up the prices by several houndred percent (src3). Trading, mining, recycling, and diposal of REEs is regulated (src2). China is also dominant in processing capabilities (src2).


\section{Environmental aspects at aquisition}

The main source of REEs is by exploiatation of mining operations (src2). The first step in aquiring REEs is finding where they are, then mining them, crushing and grinding the ores, and then different extraction techniques to get the different REEs out of the ores (src2). The processing step is the one that is the most technical, as the chemical properties of REEs make them difficult to separate from eachother (src3).


\subsection{Mining}

One issue in mining is unregulated mining operations and illigal mining operations (src2). Mining often involves significant disturbance to the land (src2). Illigal mining operations tycpically cause more damange due to complete lack of regulation and use of non-compliant mining practices (src2). One less land invasive techinique is called in-situ leaching, where the REEs are extracted from the ore without removing the ore from the ground (src2). While this method may be less damaging for the land, it does have a much higer risk of environmental damage due to the chemicals involved (src2). Mining sites can be rehabilitated post-mining (src2). In-situ leaching also produces less mining waste and can therefore also be cheaper (src2).

\subsection{Processing}

There are heavy, HREEs, and light, LREEs, REEs and one difficulty in processing is separating the two, as they have similar chemical properties, requiring intricate and precise techniques (src2). The precense of radioactive elements such as thorium and uranium adds more complexity to the extraction (src2). Different acids are often used when processing the ore, such as sulfuric acid when leaching from bastnaesite, or sodium hydroxide when leaching from monazite and xenotime (src2). The issue with using these acids is that they also dissolve other impurities, which then requires further purification steps (src2). Electrochemical processing is a more environmentally friendly technique that involves less use of chemicals (src2).

\subsection{Environmental aspects at end-of-}

Move text about e-waste here?
Landfills filled with e-waste realease toxic chemicals that pollute the soil, water, air, and ultimately human and animal health (src7?).
Heavy metals in e-waste...
RoHS ...
Basel convention ...
Illegal export ...
Other dangerous substances other than metals are plastics and flame retardants. 
Only about 9.33 million tones out of 53.6 million tones that were generated in 2019 were documented and recycled properly (src7). From refrigirators only it is estimated that 98 million tonns of CO2 equilvalents were realesd, due to the lack of proper recycling (src7). That is 0.3% of the global CO2 emissions from energy sources, or as much as the country xxx released in xxx.

\subsection{Toxicity}


\section{Defining e-waste}

E-waste, also known as Waste Electrical and Electronic Equipment (WEEE), is defined as ... 
It contains valuable material that can be recycled ...

In 2019 the global generation of e-waste, or waste electrical and electronic equipment (\textit{WEEE}), was 53.6 million mt (src6 -> 1), which is predicted to increase to 74.7 metric tons by 2030, which is not including solar panels \cite{javed2024}. Looking at only solar panels, it is predicted that they will contribute to 4-14\% of the e-waste in 2030 and over 50\% (~78 million tons) in 2050 \cite{javed2024}. The e-waste market is expected to grow to \$145 million by 2030 (src6 -> 3). Recovering REEs from e-waste is driven by environmental (avoiding new exploitation and having circular \textit{techonomy}), geopolitical (reducing dependency on single suppliers), and economical (reducing costs when traditional supply is dwindling and stabilizing global supply) factors (src6).

The world's leading countries for the generation of e-waste are China (10.1 million tons), the US (6.9 million tons), and India (3.2 million tons) \cite{javed2024}. One paper estimates that recycling Nd-Fe-B magnets from old hard drives in the US only, could supply 5.2\% of the global (excluding China) demand for those REEs (src6 -> 10).

The composition of e-waste is 65\% metals, 21\% plastics, 16\% other materials (\cite{javed2024} Take figure 6 from the paper). The metals are 50\% ferrous metals such as iron and steel, and 13\% non-ferrous metals such as copper, aluminium, precious metals, and rare-earth elements. 

1 million cell phones equal 35274 lbs of copper, 772 lbs of Ag, 75 lbs of Au, and 33 lbs of palladium \cite{javed2024}.

\section{Recycling and recovery}

The recovery of REEs from end-of-life (\textit{EOL}) products is a challenge that is gaining more attention as the demand for REEs increases \cite{USDoE2024}. Another reason for the increase in interest in recycling REEs is the environmental impact of mining and processing REEs \cite{USDoE2024}. Presently it is mostly phosphors and catalysts that are recycled \cite{britannica2024}. However, batteries such as nickel-metal hydride (\textit{NiMH}) batteries and some permanent magnets hold 25-30\% of their weight in REEs, which is more than any ore deposit \cite{britannica2024}. Recycling aligns with a circular economy approach \cite{circular2016}, but recovery is difficult for many reasons, including the mixture of materials and metals in the products and the use of glass, polymers, and other non-metals.

Traditional methods involve using chemicals to dissolve the REEs from the product, and then precipitate them out of the solution \cite{sanchez2024}. Other methods also involve using heat or electricity to extract the REEs \cite{sanchez2024}. The similarities between the different extraction techniques are that they try to exploit the different chemical and physical properties of the REEs to separate them from each other and the ore or e-waste \cite{sanchez2024}. What makes this process difficult is that the REEs are chemically and physically similar to each other \cite{britannica2024}, which requires a high degree of precision and accuracy in the separation process.

\subsection{Hydrometallurgical recycling}

Can also extract REEs rapidly with high efficiency. But has the issue of using a lot of chemicals, which can be harmful to the environment (src6 -> 14-17). This is the most used method currently for industrial scale recycling and recovery \cite{javed2024}. Two types of machines used during liquid extraction are mixer-settlers and batch extractors, but they have drawbacks such as long mixing times and large plant footprints \cite{javed2024}. Process intensification and the field of microfluidics are being explored to improve the efficiency of the process \cite{javed2024}. Microfluidic devices are used for solvent extraction. they have a smaller footprint and higher efficiency than traditional methods (src7 -> Santana et al. 2020). These microflow devices also have many other benifits over the tradition batch process (src7 -> Vural et al. 2016). (Remove this microfluid section??)

See src7 -> fig 10 for an overview of hydromeallurgical processes.

The hydrometallurgical process is a chemical process, where different chemicals, acids or bases, are used to leach, ie. dissovle, the REEs. There are different categories of hydrometallurgical processes. The main three are chemical leaching, acid/alkali leaching, and bio-hydrometallurgical approach \cite{javed2024}. 

TODO rewrite this section

\subsubsection{Chemical leaching}

The four sub-categories of chemical leaching are \textit{cyanide leaching}, \textit{thio-sulphate leaching}, \textit{thei-urea leaching}, and \textit{halide leaching} \cite{javed2024}. The latter three can all be grouped as \textit{non-cyanide leaching}. As can be expected, the major drawback of cyanide leaching is that it is highly toxic and also highly corrosive (add src). The reason for using cyanide is that it is cheap to use and effective at extracting Au and Ag (src 7).

The non-cyanide group all have the advantages that they are less toxic than cyanide (add src), less corrosive (except halide), and theio-urea is also classified as environmentally friendly (add src). Which one to choose depends on which metal there is to be extracted (add src), even though they all are mostly used to extract precious metals (add src). Thiosulfate leaching has the issue that it is not as effective as cyanide, with about 93\% leaching rate for Au, and is also more expensive \cite{javed2024}. Thiourea leaching has a 99\% leaching rate for Au and works faster \cite{javed2024}. Halide leaching includes \textit{chloride}, \textit{bromide}, and \textit{iodide} leaching, and has an Au leaching rate of around 90\% \cite{javed2024}.

\subsubsection{Acid/alkali leaching}

Acid leaching is the most common approach for metal recovery from electronic waste \cite{javed2024}. The process is well-studied, flexible, and inexpensive. Different acids are used depending on which metal is to be recovered. Within acid leaching, there are three different approaches: acid only, acid and oxidiser agent, or multi-stage acid leaching. Depending on the method used and the metal type, the recovery rate is between 6-90%. 

TODO expand more here?

\subsection{Pyrometallurgical recycling}

Also called \textit{thermal extraction}, pyrometallurgy is a method that uses high heat to extract materials by melting and incineration \cite{javed2024}. One important downside of pyrometallurgy is the hazardous fumes that are released during the process and the high energy usage required to reach the needed temperatures (src6 -> 14). It involves melting in a blast furnace, often a \textit{plasma arc furnace} or \textit{copper melter}, as well as incineration and high-heat roasting (src8). Metals can be recovered from the gas, the melted product, or from the slag. It has about a 70\% recovery rate, where it's mostly Cu that is recovered (src8 -> Cui and Rover 2011). However, pyrometallurgy does not adhere to the UN sustainability goals, as it releases dioxins, uses a high amount of energy, and can create toxic slag. Some REEs can not be recovered this way and will be lost from the circular economy by being discarded together with the toxic slag. Another drawback is that it is also a highly advanced process, as it requires an advanced control system for accurate temperature control and extraction (src8 -> Cui and Zhang 2008a, b). By combining the pyrometallurgical process with vacuum, sublimation, and distillation, more REEs such as Sb, Pb, and Bi can be recovered as well (src8 -> Flandinet et al 2012, Zhan and Xu 2009).

\subsection{Bio-hydrometallurgical approach}

The use of bacteria or fungi is gaining attention as an alternative to traditional hydro- and pyrometallurgical methods (add src). In one study it was found that the bacteria \textit{Pseudomonas chlororaphis} was able to dissolve Ag, Au, and Cu at rates of 8.2\%, 12.1\%, and 52.3\% respectively (src7 -> Ruan et al. 2014). 

Another study looked at the fungi \textit{A. niger}, also known as \textit{black mold}, and found that in a two-step chemo-bioleaching process, a recovery rate of 70.6\% of Cu was achieved (src7 -> Jadhav et al). The use of biological agents is still new, and many variables have to be controlled to make the process reliable (pH, temperature, contamination, etc.) \cite{javed2024}. 

Ionic liquids (\textit{Ils}) can potentially play a role in the future of metal extraction \cite{javed2024}. Ils are often hydrophobic and can therefore also extract other hydrophobic materials. They do however come with the downside of being toxic and having poor biodegradability. Another future candidate is deep eutectic solvents (\textit{DES}), which overcome these problems of Ils \cite{javed2024}. 

Another compound that has been explored is glycine (src7 -> Li et al. 2018). It has shown to be effective at dissolving Cu at a rate of 98\% after 48 h. However, it does struggle with solving other metals such as Au and Ag \cite{javed2024}.

Chelating, which is the chemical process of binding a metal ion to a chemical compound, has also gained popularity in recent years \cite{javed2024}. By using biodegradable chemicals such as \textit{DTPA} or \textit{NTA}, metals such as Cr, Cu, Zn, and Pb can be removed from polluted soils (src7 Chauhan et al.).



## Acid-free recycling

An alternative to hydrometallurgical recycling is acid-free recycling (src6 -> 19-21).

## Biological recycling

## Green technologies

Mentioned in src7. Include or not? membrane filtration, green adsorption, and photocatalysis.

## Cryo milling


#### Also mention:
different nuances to every technique, such as extraction in vaccuum, high pressure, different temperatures, etc. And also combinations of hydro- and pyrometallurgical methods \cite{javed2024}.

britannica2024
\section{Future outlook}

With the current demand and known sources, the reserves would last for another 900 years \cite{britannica2024}. But since the demand is increasing at a rate of around 10\% every year, the reserves would only be enough until the mid-21st century \cite{britannica2024}, or another 314 years (src5 or https://pubs.acs.org/doi/10.1021/es203518d). This is without taking recycling into account. The estimated reserves do however change due to more exploration, advancements in exploration techniques, and advancements in refining making previously uneconomical deposits profitable \cite{REETechnology}.

REEs can be divided into three groups based on their demand and supply: critical (Nd, Eu,
Tb, Dy, Y, and Er), uncritical (La, Pr, Sm, and Gd), and excessive (Ce, Ho, Tm, Yb, and Lu) (src5 -> 59).

\subsection{Alternative sources}
TODO add some text

\subssubection{Extraction from coal byproducts}

Recovery of REEs from hard coal plants in Poland \cite{USDoE2024}. Content of REEs of the coal that is being burned is between tens of ppm up to 1\% (src5 -> 9,25-47). Both fly ash and bottom ash are both byproducts of the burning of coal. Fly ash is the light particle ash that is carried upward by smoke and heat, while bottom ash is the heavy ash that is falls downward in the furnace (add src). The REE content of the bottom ashes in one study was lower than the global average throughout the earths crust, 199-286 ppm (src5). However, since it is a byproduct that is already heavily processed into powder, it could still be a viable source of REEs. Ash could be a good resouce of REEs, since it's ready available and already in a powder form which would make processing easier.

\subsubsection{Other viable sources}

Extraction from urban sources, ie. urban mining, uses 17 times less energy according to the Norwegian research institute (src7, or add 1st hand src?).

\subsubsection{Not using rare earths}

3R rule: reduce, reuse, recycle. Additonally also "redesign" (src7 -> Debnath et al. 2018)




\section{Summary}

The use of bio-leaching processes could be a good environmentally friendly alternative when they reach a higher efficiency and broader industrial usage.

\section{AI statement}

Microsoft Copilot and Grammarly have been used in the writing of this report. Both AI tools have been used to aid with grammatical errors, sentence structure, and spelling. Copilot has been used to help improve the text by asking review-like questions such as "Are there any conflicting statements in the text?" and "Is the text clear and concise?". It was also used to structure the references into the correct \LaTeX format. Grammarly has been used to help with spelling, grammar, and to suggest alternative words or phrases. Both tools have been used to help improve the quality of the text and to make sure that the text is clear and accurate. The final text reflects the author's views, opinions, and knowledge.



\printbibliography

\end{document}
