
\section{Recycling and recovery}

The recovery of REEs from end-of-life (EOL) products is a challenge that is gaining more attention as the demand for REEs increases (src1). Another reason for the increase in interest in recycling REEs is the environmental impact of mining and processing REEs (src1). Presently it is mosly phosphors and catalysts that are recycled (src3). However, batteries such as nickel-metal hydrite batteries and some permanent magnets holds 25-30\% of their weight in REEs, which is more than any ore deposit (src3). Recycling aligns with a circular economy approach (src2). Recovery is difficult for many reasons, including mixture of materials and metals in the products, use of glass, polymers, and other non-metals.

Traditional methods involve using chemicals to dissolve the REEs from the product, and then precipitate them out of the solution (add src). Other methods also involve using heat or electricity to extract the REEs (add src). The similarities between the different extraction technicues is that they try to exploit the different chemical and physical properties of the REEs to separate them from eachother and the ore or electronical waste (add src). What make this process difficult is that the REEs are chemically very similar to eachother (src3), which requires a high degree of precision and accuracy in the separation process (add src).

In 2019 the global generation of e-waste, or waste electrical and electronic equipment (WEEE), was 53.6 million tons (src6 -> 1), which is predicted to increase to 74.7 metric tons by 2030, which is not including solar panels (src7). Looking at only solar panels, it is predicted that they will contronbute to 4-14\% of the e-waste in 2030 and over 80\% (~78 million tons) in 2050 (src7). The market for e-waste is expected to grow to \$145 million by 2030 (src6 -> 3). Recovering REEs from e-waste is driven by environmental (avoiding new exploitation and having circular techeconomy), geopolitical (reducing dependency on single suppliers), and economical (reducing costs when traditional supply is dwindling and stabilizing global supply) factors (src6).

One paper estimates that recycling Nd-Fe-B magnets from old HDDs in the US only, could supply 5.2\% of the global (excluding China) demand for those REEs (src6 -> 10).

The world leading countires for generation of WEEE are Chine (10.1 million tons), the US (6.9 million tons), India (3.2 million tons) (src7).

The Basel convention prohibits the export of hazardous waste from developed to developing countries (add src). Therefore e-aste management has become a large area of research (src7). Switzerland and the Netherlands are the leading countries in e-waste management (src7 -> Bhutta et al. 2011).

### Defining e-waste

E-waste, also known as Waste Electrical and Electronic Equipment (WEEE), is defined as ... 
It contains valuable material that can be recycled ...
Can be divided into 54 distinct groups (src7 -> Zhang et al. 2013, Nakamura et al. 2015, forti et al. 2020), or into 6 broad catagories based on the type of equipment (src7).

Composition of E-wate is 65\% metals, 21\% plastics, 16\% other materials (src7. Take figure 6 from the paper). The metals are 50\% ferrous metals such as iron and steel, and 13\% non-ferrous metals such as copper, aluminum, precious metals, and rare-earth elements. 

1 million cell phones equals 35274 lbs of copper, 772 lbs of silver, 75 lbs of gold, and 33 lbs of palladium (src7), with a market value of xxx.


## Chemical recycling

## Hydrometallurgical recycling

Can also extract REEs rapidly with high efficiency. But has the issue of using a lot of chemicals, which can be harmful to the environment (src6 -> 14-17). This is the most used method currently for industrial scale recycling and recovery (src7). Two types of machines used during liquid extraction are mixer-settlers and batch extractors, but they have drawbacks such as long mixing times and large plant footprints (src7). Process intensification and the field of microfluidics are being explored to improve the efficiency of the process (src7). Microfluidic devices are used for solvent extraction. they have a smaller footprint and higher efficiency than traditional methods (src7 -> Santana et al. 2020). These microflow devices also have many other benifits over the tradition batch process (src7 -> Vural et al. 2016).

See src7 -> fig 10 for an overview of hydromeallurgical processes.

The hydrometallurgical process is a chemical process, where different chemicals, acids or bases, are used to leach, ie. dissovle, the REEs. There are different categories of hydrometallurgical processes. The main three are chemical leaching, acid/alkali leaching, and bio-hydrometallurgical approach (src7). 

### Chemical leaching

The four sub-categories of chemical leaching are cyanide leaching, thio-sulphate leaching, thei-urea leaching, and halide leaching (src7). The latter three can all be grouped as non-cyanide leaching As can be expected, the major drawback of cynaide leaching is that it is highly toxic and also corrosive (add src). The reason for using cyanide is that it is cheap to use and effective at extracting gold and silver (src 7).

The non-cyanide group all have the advantages that they are less toxic than cyanide (add src), less corrosive (except halide), and theio-urea is also classified as environmentally friendly (add src). Which one to choose depends on which metal there is to be extracted (add src), even though they all are mostly used to extract precious metals (add src). Thiosulfate leaching has the issue that it is not as effective as cyanide, about 93\% leaching rate for Au, and also more expensive (src7). Thiourea leaching has a 99\% leaching rate for Au and faster kinetics (?) (src7). Halide leaching includes chloride, bromide, and iodide leaching, and has an Au leaching rate of around 90% (src7).

### Acid/alkali leaching

Acid leaching is the most common approach for metal recovery from electronic waste (src7). the process is well studied, flexible, and inexpensive. Different acids are used depending on which metal is to be recovered. Within acid leaching there are three different approaches: acid only, acid and oxidiser agent, or multi-stage acid leacing. Depending on the method used and the metal type, the recovery rate is between 6-90%. 

### Bio-hydrometallurgical approach ...

The use of bacteria or fungii is gaining attention as an alternative to traditional hydro- and pyrometallurgical methods (add src). In one study it was found that the bacteria Pseudomonas chlororaphis was able to disolve Ag, Au, and Cu at rates of 8.2\%, 12.1\%, and 52.3\% respectively (src7 -> Ruan et al. 2014). Another study looked at the fungii A. niger, also known as black mold, and found that in a two-step chemo-bioleaching process, a recovery rate of 70.6\% of Cu was acheived (src7 -> Jadhav et al). The use of biological agents is still new, and there are many variables that has to be controlled to make the process reliable (pH, temperature, contamination, etc.) (src7). Ionic liquids (Ils) can potentially play a role in te future of metal extration (src7). Ils are often hydrophobic and can therefore also extract other hydrophobic materials. They do however come with the downside of being toxic and having poor biodegradability. Another future candidate are deep eutectic solvents (DES), which are overcome these problems of Ils (src7). Another compound that has been explored is glycine (src7 -> Li et al. 2018). It has shown to be effective at dissolving Cu at a rate of 98\% after 48 h. But it does struggle with solving other metals such as Au and Ag (src7).

Chelating, which is the chemical process of binding a metal ion to a chemical compound, has also gained popularity in recent years (src7). By using biodegradable chemicals such as DTPA or NTA, metals such as Cr, Cu, Zn, and Pb can be removed from polluted soils (src7 Chauhan et al.).
## Pyrometallurgical recycling

Even though it has a high and rapid recovery rate of REEs, it also has issues with high energy use (src6 -> 14). It is a method that uses high heat, themal extraction, of materials by melting and incineration (src7). One important downside of pyrometallurgy is the hazardous fumes that are released during the process. It involves melting in a blast furnace, plasma arc furnace, or copper melter, as well as incineration and high heat roasting (src8). Metalls can be recovered from the gas. It has about an 70\% recovery rate (src8 -> Cui and Rover 2011). Does not adhere to the UN sustainability goals, as it releases dioxins and uses a high amount of energy, and can create toxic slag. Advanced as it requires a advanced control system (costly) (src8 -> Cui and Zhang 2008a, b). Mostly Cu is recovered. Some metals can not be recovered, leading to losses of potentially recoverable metals.

By using a vaccuum pyrometallurgical approach, REEs can be recovered with the help of sublimation and distillation (src8). This process is better at recovering heavy metals such as Sb, Pb, and Bi (src8 -> Flandinet et al 2012, Zhan and Xu 2009).

## Acid-free recycling

An alternative to hydrometallurgical recycling is acid-free recycling (src6 -> 19-21).

## Biological recycling

## Green technologies

Mentioned in src7. Include or not? membrane filtration, green adsorption, and photocatalysis.

## Cryo milling


#### Also mention:
different nuances to every technique, such as extraction in vaccuum, high pressure, different temperatures, etc. And also combinations of hydro- and pyrometallurgical methods (src7).