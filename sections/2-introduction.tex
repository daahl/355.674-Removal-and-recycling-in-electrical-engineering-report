
\section{Introduction}
The main focus of this report will be on the environmental aspects and the recycling of REEs. The following chapter will therefore introduce REEs for individuals who are only vaguely or not at all familiar with REEs. As electronic waste, also called \textit{e-waste}, and recycling of REEs are tightly connected, there will also be a short introduction to the topic of e-waste. 

\subsection{What are REEs?}

Metals are usually divided into base, precious, and rare earth elements \cite{javed2024}. REEs are a group of 17 elements from the periodic table \cite{USDoE2024}. They were first discovered in the 18th century in a village called \textit{Ytterby} located in Sweden, by army lieutenant Carl Axel Arrhenius \cite{britannica2024}. Four elements in the periodic table have been named after Ytterby, namely \textit{yttrium}, \textit{ytterbium}, \textit{terbium}, and \textit{erbium} \cite{sciencehistory2024}. The first REE to be isolated was \textit{cerenium} in 1803 \cite{britannica2024}.  

\subsection{Why they are called \textit{rare}}

The word \textit{rare} in \textit{rare earth elements} comes from the historical name, when it was once though to be rare \cite{sciencehistory2024}. REEs belong to the 50th percentile of most abundant elements in the earth's crust \cite{britannica2024}. This means that they are more abundant than e.g. diamonds \cite{diamonds} in the earth's crust. REEs are distributed throughout the earth's crust, but due to their geological properties, they are dispersed mostly in low quantities, too low for there to be any economic value in mining and refining them \cite{REEDistribution}. The average quantity throughout the earth's crust is 60-70 ppm \cite{REEDistribution}. Most of the time, REEs have to be mined together with other minerals for there to be any economic value in mining them \cite{REEDistribution}. 

\subsection{Applications of REEs}

The use cases for REEs are wide, from the probably most well-known being neodymium magnets \cite{USDoE2024} to lesser-known ones such as mirrors \cite{britannica2024}. The many different magnetic \cite{USDoE2024}, electric \cite{britannica2024}, thermal \cite{britannica2024}, and catalytic \cite{britannica2024} properties make them usable in many areas. They have a role in applications within national security, as many modern defence systems rely on the use of REEs \cite{USDoE2024}. They are also used in the production of batteries, phosphors, and catalysts \cite{USDoE2024}. Their use in these low-level applications makes them a critical part of many modern technologies, in everything from healthcare and transportation to energy generation and consumer electronics \cite{USDoE2024}. In electronics, between 0.1-5\% of the total weight of the product is REEs. But for magnets, this can be as high as 25\% \cite{britannica2024}. REEs have also found new uses in additive manufacturing, where they have been shown to increase the durability of the materials \cite{additive2022}. REEs can also be used in cement, where they improve durability and therefore reduce the need for repair and replacement \cite{cement2009}. REEs can be used in cleaning soil, in a similar way as they're used in cleaning exhaust gases, by binding to heavy metals and removing them from the soil \cite{soil2017b}. For the same reason, they can also be used to filter water \cite{water2024}.