
\section{Introduction}

The following chapter will introduce REEs for individuals who are only vaguely or not at all familiar with them. There will be a short introduction to e-waste as well, as recovery of e-waste and REEs are tighly connected. The main focus of this report will be on the enviromental aspects and the recycling of REEs. Metals are usually divided into base, precious, and REE (src7).

\subsection{What are REEs?}

REEs are a group of 17 elements from the periodic table (src1). They were first dicovered in the 18th century in a village called Ytterby, in Sweden, by army lieutenant Carl Axel Arrhenius (src3). The first REE to be isolated was cerenium, in 1803 (src3).

\subsection{Why they are called \textit{rare}}

Contrary to what one might expect, the word "rare" in "Rare Earth Elements" does not come from their rarity in the earths crust, as is the case with ie. diamonds (add src), but rather from their industrial availability (add src). REEs belong to the 50th percentile of most abundant elements in the earths crust (src3). REEs are distributed all throughout the earths crust, but due to their geological(?) properties they are dispersed mostly in low quantities, too low for there to be any economical value in mining and refining them (add src). The average quantity throught the earths crust is XXX (add src), while the lowest quantitiy that is mined (mining purely for REEs) is XXX (add erc). Most of the time, REEs have to be mined together with other minerals for there to be any economical gain (add src). Both EU and US have REEs on their lists of critical raw materials (src8 -> EU 2014, US 2018).

\subsection{Uses of REEs}

The use cases for REEs are wast, from the probably most well known being neodymium magnets (src1) to lesser known ones such as mirrors (src3). They also have a role in applications within national security, as many modern defence systems rely on the use of REEs (src1). They are also used in the production of batteries, phosphors, and catalysts (src1). Their use in these low-level applications make them a critical part of many modern technologies, in anything from healthcare and transportation, to energy generation and consumer electronics (src1). In electronic, between 0.1-5\% of the total weight of the product is REEs, but for magnets this can be as high as 25\% (src3). REEs can has also found new uses in additive manufacturing, where they have shown to increase the durability of the materials (src2). REEs can also be used in cement, where it improves the durability and therefore reduces the need for repair and replacement (src2). REEs can be used in cleaing soil, in a similar way as they're used in cleaning exhaust gases, by binding to the heavy metals and removing them from the soil (src2). For the same reason they can also be used to filter water (src2).

In the 1950s, the world wide production of ROEs, which are REE oxides, was below 5000 Mg (src5). In 2023, the production was up 70 fold to 350'000 Mg (src5). Recent forecasts predict that the demand will increase to 450'000 Mg by 2035 (src5 or www.mdpi.com/2071-1050/16/5/1951).

\subsection{The function of REEs}

REEs are similar in some senses and different in others (add src). REEs are used, among others, for their magnatic (src1), catalytic (add src), and thermal (add src) properties.