\section{Future outlook}

REEs can be divided into three groups based on their demand and supply: critical (Nd, Eu, Tb, Dy, Y, and Er), uncritical (La, Pr, Sm, and Gd), and excessive (Ce, Ho, Tm, Yb, and Lu) \cite{coalPoland}. With the current demand and known sources, the REE reserves would last for another 900 years \cite{britannica2024}. But since the demand is increasing at a rate of around 10\% every year, the reserves would only be enough until the mid-21st century \cite{britannica2024}, or another 314 years \cite{coalPoland}. This is without taking recycling into account. The estimated reserves do however change due to more exploration, advancements in exploration techniques, and advancements in refining making previously uneconomical deposits profitable, and of course by improvements in recycling \cite{REETechnology}.

\subsection{Alternative sources}

Outside of traditional mining and recycling sources, there are other sources of REEs that are being researched. A short overview of some of these sources are presented below.

\subsubsection{Extraction from coal byproducts}

Recovery of REEs from hard coal plants in Poland \cite{USDoE2024}. Content of REEs of the coal that is being burned is between tens of ppm up to 1\% \cite{coalPoland}. Both fly ash and bottom ash are both byproducts of the burning of coal. Fly ash is the light particle ash that is carried upward by smoke and heat, while bottom ash is the heavy ash that is falls downward in the furnace \cite{ashPalmer}. The REE content of the bottom ashes in one study was lower than the global average throughout the earths crust, 199-286 ppm \cite{coalPoland}. However, since it is a byproduct that is already heavily processed into powder, it could still be a viable source of REEs. Ash could be a good resouce of REEs, since it's ready available and already in a powder form which would make processing easier.

\subsubsection{Extraction from urban sources}

Extraction from urban sources, i.e., urban mining, uses 17 times less energy according to the Norwegian research institute \cite{javed2024}. The concentration of REEs are higher in urban sources (i.e. garbage dumps and landfills) than in traditional mines \cite{javed2024}. These \textit{urban mines} would not only be a good source of REEs, but by processing the disposed electronics, it would also reduce the amount of e-waste that is leaching toxic chemicals into the environment \cite{javed2024}.

\subsubsection{Not using rare earths}

One alternative to using REEs is to not use them at all. The previously well known \textit{3R rule} - reduce, reuse, recycle - can be expanded to also include \textit{redesign}. Redesigning can include making it easier to separate and recycle precious metals and REEs from e-waste, but also to design products that do not require REEs at all \cite{motorNoREE}.


