britannica2024
\section{Future outlook}

With the current demand and known sources, the reserves would last for another 900 years \cite{britannica2024}. But since the demand is increasing at a rate of around 10\% every year, the reserves would only be enough until the mid-21st century \cite{britannica2024}, or another 314 years (src5 or https://pubs.acs.org/doi/10.1021/es203518d). This is without taking recycling into account. The estimated reserves do however change due to more exploration, advancements in exploration techniques, and advancements in refining making previously uneconomical deposits profitable \cite{REETechnology}.

REEs can be divided into three groups based on their demand and supply: critical (Nd, Eu,
Tb, Dy, Y, and Er), uncritical (La, Pr, Sm, and Gd), and excessive (Ce, Ho, Tm, Yb, and Lu) (src5 -> 59).

\subsection{Alternative sources}
TODO add some text

\subssubection{Extraction from coal byproducts}

Recovery of REEs from hard coal plants in Poland \cite{USDoE2024}. Content of REEs of the coal that is being burned is between tens of ppm up to 1\% (src5 -> 9,25-47). Both fly ash and bottom ash are both byproducts of the burning of coal. Fly ash is the light particle ash that is carried upward by smoke and heat, while bottom ash is the heavy ash that is falls downward in the furnace (add src). The REE content of the bottom ashes in one study was lower than the global average throughout the earths crust, 199-286 ppm (src5). However, since it is a byproduct that is already heavily processed into powder, it could still be a viable source of REEs. Ash could be a good resouce of REEs, since it's ready available and already in a powder form which would make processing easier.

\subsubsection{Other viable sources}

Extraction from urban sources, ie. urban mining, uses 17 times less energy according to the Norwegian research institute (src7, or add 1st hand src?).

\subsubsection{Not using rare earths}

3R rule: reduce, reuse, recycle. Additonally also "redesign" (src7 -> Debnath et al. 2018)


