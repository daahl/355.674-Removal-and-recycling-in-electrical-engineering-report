
\section{Global aspects}

The following chapter will cover the global aspects of REEs, such as the world wide yearly usage, economic value, global branch distribution, how and where they are mined, and political aspects. As the aquisition of REEs is a global issue, it is heaviliy dependent on geopolitical and  geoecomical factors (src2).

\subsection{World wide yearly usage and supply}

The use of REEs is many sectors and coutries. In the graph/table below is an overview (TODO). China is believed to have a black market for REEs, which is not included in the official numbers, but are believed to add an additional 25\% to their production (src 3). Demand for REEs increase at a rate of 10\% every year. Both the EU and the US have REEs on their lists of critical raw materials (src8 -> EU 2014, US 2018).

\subsection{Economic value}

For there to be any econmical value in mining for REEs, the concentration has to be 5\% or higher (src 3). If mined together with other minerals, it can be economically viable to mine down to 0.5\% concentrations (src 3). Advancements in aquisition and recycling is also influencing the price (src2). The price is influenced by export regulations, tariffs, and trade agreements (src2).

\subsection{Global branch distribution}

The push for decarbonisation has increased the demand for REEs, as they are a critical component in electric vehicles and wind turbines (src2).

\subsection{Where are they acquired?}

REEs are mostly aquired through mining (src3, src2). Secondary resource is recycling (src2). There are currently 160 mineral sources of REEs, however, only 4 of them are mined: bastnasite, laterite, monazite, and loparzite (src3). There are also other non-mineral sources of REEs, such as deep-sea clays outside of the Minamitori Islands in Japan, but they have not yet been exploited (src3). 80\% of the global supply is mined by China, mostly in the Inner Mongolia region (src 3). However, extraction takes place in many countries (src3). Other major deposits world wide are US, Canada, Australia, Russia, South Africe, south-east Asia, and India (src3).

\subsection{Political aspects}

Since China is the worlds largest producer of REEs, with 80\% (105'000 tons of RE oxide) of the global supply produced in 2017 (src3), and 70\% of the market share in 2024 (src2), there are reasons for mentioning the political aspects of REEs (src3). In the years 2009-2011 China could increase their profits on REEs by simply imposing export quotas and therefore drive up the prices by several houndred percent (src3). Trading, mining, recycling, and diposal of REEs is regulated (src2). China is also dominant in processing capabilities (src2).