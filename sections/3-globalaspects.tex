
\section{Global aspects}

The following chapter will cover the global aspects of REEs, such as the worldwide yearly usage, economic value, global branch distribution, how and where they are mined, and political aspects. As the acquisition of REEs is a global issue, it is heavily dependent on geopolitical and geoeconomical factors \cite{politics2022}.

\subsection{Worldwide yearly usage and supply}

In the 1950s, the worldwide production of \textit{ROEs}, which are REE oxides, was below 5000 mt \cite{REEPrediction}. In 2023, the production was up 70-fold to 350'000 mt \cite{REEPrediction}. Recent forecasts predict that the demand will increase to 450'000 mt by 2035 \cite{REEPrediction}.

The use of REEs is many sectors and coutries. In the graph/table below is an overview (TODO). China is believed to have a black market for REEs, which is not included in the official numbers, but are believed to add an additional 25\% to their production (src 3). Demand for REEs increase at a rate of 10\% every year. Both the EU and the US have REEs on their lists of critical raw materials (src8 -> EU 2014, US 2018).

\subsection{Economic value}

For there to be any economic value in mining for REEs, the concentration has to be 5\% or higher (src 3). If mined together with other minerals, it can be economically viable to mine down to 0.5\% concentrations (src 3). Advancements in acquisition and recycling are also influencing the price \cite{REELandscape}. The price is influenced by export regulations, tariffs, and trade agreements \cite{politics2022}.

\subsection{Global branch distribution}

The push for decarbonization has increased the demand for REEs, as they are a critical component in e.g. electric vehicles and wind turbines \cite{windturbines2022}.

\subsection{Where are they acquired?}

While REEs are to some degree acquired through recycling, the main acquisition pathway is predominantely mining \cite{britannica2024}\cite{REELandscape}. There are currently 160 mineral sources of REEs, however, only 4 of them are mined: \textit{bastnasite}, \textit{laterite}, \textit{monazite}, and \textit{loparzite} \cite{britannica2024}. There are also other nonmineral sources of REEs, such as deep-sea clays outside of the \textit{Minamitori Islands} in Japan, but they have not yet been exploited \cite{britannica2024}. 80\% of the global supply is mined by China, mostly in the Inner Mongolia region (src 3). However, extraction takes place in many countries \cite{britannica2024}. Other major deposits worldwide are US, Canada, Australia, Russia, South Africa, Southeast Asia, and India \cite{britannica2024}.

\subsection{Political aspects}

China is the world's largest producer of REEs, largely because they have the world's largest refining capabilities \cite{ChinaRefining}, with 80\% (105'000 mt of RE oxide) of the global supply produced in 2017 \cite{britannica2024}, and 70\% of the market share in 2023 \cite{ChinaRefining}. Due to this there are reasons for mentioning the political aspects of REEs \cite{britannica2024}. One example where China used their dominance as a suplier was in the years 2009-2011, where China imposed export quotas and drove up the market price of REEs by several hundred per cent\cite{britannica2024}. Trading, mining, recycling, and disposal of REEs are regulated \cite{REELandscape}.