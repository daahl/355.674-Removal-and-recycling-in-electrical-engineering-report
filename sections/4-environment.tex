
\section{Environmental aspects at aquisition}

The main source of REEs is by exploitation of mining operations (src2). The first step in aquiring REEs is finding where they are, then mining them, crushing and grinding the ores, and then different extraction techniques to get the different REEs out of the ores (src2). The processing step is the one that is the most technical, as the chemical properties of REEs make them difficult to separate from eachother (src3).


\subsection{Mining}

One issue in mining is unregulated mining operations and illigal mining operations (src2). Mining often involves significant disturbance to the land (src2). Illigal mining operations tycpically cause more damange due to complete lack of regulation and use of non-compliant mining practices (src2). One less land invasive techinique is called in-situ leaching, where the REEs are extracted from the ore without removing the ore from the ground (src2). While this method may be less damaging for the land, it does have a much higer risk of environmental damage due to the chemicals involved (src2). Mining sites can be rehabilitated post-mining (src2). In-situ leaching also produces less mining waste and can therefore also be cheaper (src2).

\subsection{Processing}

There are heavy, HREEs, and light, LREEs, REEs and one difficulty in processing is separating the two, as they have similar chemical properties, requiring intricate and precise techniques (src2). The precense of radioactive elements such as thorium and uranium adds more complexity to the extraction (src2). Different acids are often used when processing the ore, such as sulfuric acid when leaching from bastnaesite, or sodium hydroxide when leaching from monazite and xenotime (src2). The issue with using these acids is that they also dissolve other impurities, which then requires further purification steps (src2). Electrochemical processing is a more environmentally friendly technique that involves less use of chemicals (src2).

\subsection{Environmental aspects at end-of-}

Move text about e-waste here?
Landfills filled with e-waste realease toxic chemicals that pollute the soil, water, air, and ultimately human and animal health (src7?).
Heavy metals in e-waste...
RoHS ...
Basel convention ...
Illegal export ...
Other dangerous substances other than metals are plastics and flame retardants. 
Only about 9.33 million tones out of 53.6 million tones that were generated in 2019 were documented and recycled properly (src7). From refrigirators only it is estimated that 98 million tonns of CO2 equilvalents were realesd, due to the lack of proper recycling (src7). That is 0.3% of the global CO2 emissions from energy sources, or as much as the country xxx released in xxx.

\subsection{Toxicity}