
\section{Environmental aspects at aquisition}

The main source of REEs is by exploitation of mining operations \cite{REELandscape}. The first step in acquiring REEs is finding deposits of high concentration, mining the ores followed by crushing and grinding, and then different extraction techniques to get the different REEs out of the powdered ore \cite{REELandscape}. The processing step is the most technical one, as the chemical properties of REEs make them difficult to separate from each other \cite{britannica2024}.


\subsection{Mining}

One issue in mining is unregulated mining operations and illegal mining operations, especially since mining already involves significant disturbance to the land \cite{unregulatedMining}. Illegal mining operations typically cause more damage due to a complete lack of regulation and the use of non-compliant mining practices \cite{unregulatedMining}. One less land invasive technique is called \textit{in-situ leaching}, where processing chemicals are pumped through the deposits to extract the REEs without removing the ore from the ground \cite{REELandscape}. While this method may be less damaging for the land, it does have a much higher risk of environmental damage due to the chemicals involved \cite{unregulatedMining}. This method is often used because it's cheaper and produces less mining waste \cite{unregulatedMining}.

\subsection{Processing}

There are heavy REEs, called \textit{HREEs}, and light REEs, called \textit{LREEs}, and one difficulty in processing is separating the two, as they have similar chemical properties, requiring intricate and precise techniques \cite{HREELREE}. The presence of radioactive elements such as \textit{thorium} and \textit{uranium} adds more complexity to the extraction \cite{radioactive}. Different acids are often used when processing the ore, such as \textit{sulfuric acid} when leaching from bastnaesite, or \textit{sodium hydroxide} when leaching from monazite and xenotime \cite{REELandscape}. The issue with using these acids is that they also dissolve other impurities, which then requires further purification steps \cite{radioactive}. Electrochemical processing is a more environmentally friendly technique that involves fewer chemicals \cite{radioactive}.

\subsection{Environmental aspects at end-of-life}

Landfills filled with e-waste release toxic chemicals such as flame retardants, heavy metals such as mercury, REEs, and other materials such as plastics that pollute the soil, water, air, and ultimately human and animal health \cite{javed2024}. The Basel Convention prohibits the export of hazardous waste from developed to developing countries (add src). Therefore e-waste management has become a large area of research \cite{javed2024}. Switzerland and the Netherlands are currently the leading countries in e-waste management (src7 -> Bhutta et al. 2011).
Only about 9.33 million mt out of 53.6 million mt that were generated in 2019 were documented and recycled properly \cite{javed2024}. From refrigerators only it is estimated that 98 million mt of $CO_2$ equivalents were released due to the lack of proper recycling \cite{javed2024}. That is 0.3\% of the global $CO_2$ emissions from energy sources, or as much as the country xxx released in xxx. TODO

